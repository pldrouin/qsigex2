\chapter*{Preface}
As a summer student working on the Sudbury Neutrino Observatory (SNO) experiment
at Carleton University, I have been asked in the second half of summer 2002 to
write a C++ package that would allow to extract neutrino fluxes from the
detector data. The package had to be designed such that it could be expanded by
the users and that its structure would be flexible enough to allow analysis
changes without code modifications. Under the supervision of Alain Bellerive, I
have written what became QSigEx 1.00. The package met the first design rule.
However, due to the limited time we had to write some important blocks of code,
the program was not as flexible and as user-friendly as it should.

In summer 2003, I decided to redesign the package. I wanted to increase the
flexibility of QSigEx, but also to incorporate more the ROOT classes into the
code. I've decided to store the information produced by the package into a ROOT
\texttt{TDirectory} structure, that would give to QSigEx a very user-friendly interface
in comparison to C-style arrays used in version 1.00. Finally, after three
months of work, I got a package that satisfied our objective.

I hope QSigEx will suit your needs!

\begin{center}
Pierre-Luc Drouin\\
Ottawa, August 2003.
\end{center}

\vspace{\stretch{1}}
\section*{WEB}

\texttt{www.physics.carleton.ca/research/sno/anal/software/qsigex.html}

\vspace{\stretch{1}}
\section*{Authors and Contributors}
\begin{description}
\item[Alain Bellerive:] Supervisor, librarian (2004), gaussian correlations
\item[Mark G. Boulay:] Contribution to QSigEx behaviour
\item[Pierre-Luc Drouin:] Code design and implementation
\item[Darren Grant:] Librarian (2002,2003), extended likelihood function coding
\item[Kathryn Miknaitis:] Librarian (2002,2003), contribution to \texttt{QSigExFit} class coding
\item[Osama Moussa:] Contribution to \texttt{QSigExTHOps} class coding
\item[Ryan MacLellan:] Contribution to \texttt{QSigExDis} and \texttt{QSigExIO} behaviour
\end{description}
