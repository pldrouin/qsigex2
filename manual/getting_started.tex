\chapter{Getting Started}

\section{QSigEx and ROOT}
QSigEx relies on ROOT \cite{root}, the object-oriented data analysis framework developed at CERN.
Not only the functions used to generate the PDFs and the object containing the sample data are ROOT objects; the QSigEx classes are themselves derived from ROOT classes and their outputs are ROOT objects as well.
It is important that QSigEx users be familiar with ROOT before using QSigEx, considering how QSigEx is based on this program.

\section{Main Modules} \label{gs:mm}
As a QSigEx user, you will have to use a reduced number of classes.
Among these classes are the QSigEx main modules.
Each of these modules accomplishes a specific task in the parameters fitting process.
Here are the steps that are followed in order to evaluate the population parameters as they are divided via the main modules.

Module:
\begin{enumerate}

\item\label{gs:mm:cuts} Since the variables $x_i$ of the population entries generally have values that are contained in a certain range and since it is often impossible in practice to predict the shape of their distribution for an infinite range of values, it's usually needed to define {\em cuts} that define regions of valid variables values.

\item\label{gs:mm:cdata} A {\em data sample} must be loaded in QSigEx.
The defined cuts are applied on this sample to get a {\em clean data sample}.

\item\label{gs:mm:pdfs} The {\em marginal PDFs} are produced by QSigEx using the ROOT objects provided by the user.
These objects are usually generated using an analytical or Monte Carlo method.
The marginal PDFs are normalized by QSigEx according to the defined cuts.

\item\label{gs:mm:cor} If the variables $x_i$ cannot be considered independent, information related to their correlations is loaded by QSigEx.

\item\label{gs:mm:mpd} Using the clean data sample and the marginal PDFs, the probability density of each data event to have a value of $x_i$ for its coordinate $i$ if the event belongs to group $G_j$ is computed for each species.

\item\label{gs:mm:jpd} Using the probability densities computed at the preceding step, the joint probability density of each data event to have coordinates $\vec{x}$ if the event belongs to species $G_j$ is computed for each species.
In the simple case where the variables $x_i$ are not correlated, this is simply the product of all marginal probability densities of a given group for each of these species, as explained in section \ref{intro}. \label{jointprobs} 

\item\label{gs:mm:fit} Finally, the parameters are evaluated, using a minimization function defined by the user and the joint probability densities computed at step \ref{jointprobs}.
\end{enumerate}

Figure \ref{gs:ex} shows an example of a standard run of QSigEx for correlated variables.
You can see the seven QSigEx classes instances that are created and for which a \texttt{Get()} member function is called.
The interface of QSigEx will be explained in more details in the next section.

\begin{figure}
\centering
\begin{boxedverbatim}
TFile f1("qsigex.root","NEW");

QSigExCuts cuts(&f1,"cardfile.dat"); 
cuts.Get();

QSigExCleanData cleandata(&f1,"cardfile.dat");
cleandata.Get();

QSigExTTreePDF pdfs(&f1,"cardfile.dat");
pdfs.Get();

QSigExGaussCor gcor(&f1);
gcor.Get();

QSigExProbs<Float_t> probs(&f1);
probs.Get();

QSigExGCJointProbs gcjprobs(&f1); 
gcjprobs.Get();

QSigExFit fitter(&f8,QExtendedLikelihood,"cardfile.dat");
fitter.Get();

f1.Write();
f1.Close();

\end{boxedverbatim}


\caption{Example of QSigEx usage}
\label{gs:ex}
\end{figure}

 

