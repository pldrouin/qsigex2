\chapter{Introduction} \label{intro}
QSigEx is a package that can be used to estimate population parameters using a
sample of events. In particular, it estimates in which proportions some event
groups are present in the population. This situation arises often in particle
physics in what is called signal extraction, from which QSigEx borrows its name. 

Let a population be composed of $n$ groups. Each entry in the population has
some measurable characteristics, such that this entry can be described by a
vector $\vec{x}$, where each $x_i$ coordinate $(i=1,2,\ldots,m)$ corresponds to
a specific observable. The population entries of a given group have
characteristic $\vec{x}$ values that allow to differenciate the group from
another one. Then, the entries in a group have a particular distribution. In
some cases, it is possible to predict the distribution associated which each
group (using Monte Carlo method for example). Provided these distributions, the
probability density of an event member of group $G_j$ to have coordinates
$\vec{x}$ is given by $f(\vec{x}|G_j)$. In QSigEx, the function $f(\vec{x}|G_j)$
is called the {\em joint probability density function} for the group $G_j$,
since it gives the probability density of an event considering all its
characteristics ($\vec{x}$).

In the case where all the groups are mutually exclusive (an entry member of
group $G_j$ cannot be member of group $G_k$ at the same time, for all $j$ and
$k$, $j\ne k$), the probability density of an entry with coordinates $\vec{x}$
to be a member of any group is given by:
\begin{equation}
f(\vec{x})=\sum_{j=1}^n f(\vec{x}\cap G_j)=\sum_{j=1}^n p(G_j)f(\vec{x}|G_j)
\end{equation}
where $p(G_j)$ is the probability of an event to be member of group $G_j$. The
set of probabilities $p(G_1),\ p(G_2),\ \ldots\ ,\ p(G_n)$, or some values
function of these probabilities (depending on the minimization function that is
defined by the user) are the population parameters that are estimated by QSigEx.
They can be viewed as the fraction of events which belong to a given group. 

In a given group, the variables $x_i$ can be independent or not. When the
variables are considered to be independent, the function $f(\vec{x}|G_j)$
$(j=1,2,\ldots,n)$ can be expressed using the {\em marginal probability density
functions} $f_i(x_i|G_j)$ for each of these variables:
\begin{equation} \label{i:jpdfvsmpdf}
f(\vec{x}|G_j)=\prod_{i=1}^{m} f_i(x_i|G_j)
\end{equation}
In this expression, $m$ is the number of dimensions of vector $\vec{x}$ used to
fit the parameters. 

Sometimes, only a pair of variables are significantly correlated. In this
situation, the user can choose to produce, using a method of his choice, a
probability density function (PDF) $f_{k,\ l}\left((x_k,x_l)|G_j\right)$ $\left(k,\ l\ \ \in
\{1,2,\ldots,m\}\right)$ that would be equivalent to $f_k(x_k|G_j)f_l(x_l|G_j)$
if the variables were not correlated. In QSigEx, this is called a {\em
bidimensional marginal probability density function}, since it doesn't take into
account all the variables. The {\em joint probability density function}
$f(\vec{x}|G_j)$ is then given by:
\begin{equation} \label{i:jpdfvsmpdf2}
f(\vec{x}|G_j)=f_{k,\ l}\left((x_k,x_l)|G_j\right)\prod_{\substack{
i=1\\
i\ne k,\ i\ne l
}}^{m} f_i(x_i|G_j)
\end{equation}
The procedure can be repeated with other pairs of variables, if these pairs are
not correlated to the other variables and also with triplet of variables if
necessary.
 
When the correlations between the variables are generally strong enough to be
considered, the information provided by the functions $f_i(x_i|G_j)$ is not
sufficient and the joint probability density functions $f(\vec{x}|G_j)$ must be
defined using more sophisticated methods. QSigEx can handle uncorrelated or
correlated variables~\cite{Karlen:1998}. It can compute joint probability densities
using either unidimensional or multidimensional marginal probability density functions.

